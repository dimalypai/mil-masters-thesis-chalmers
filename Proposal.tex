\documentclass{article}
\usepackage{fontspec}
\setmainfont{Palatino} % Georgia
\setlength{\parskip}{6pt}

\title{Master's Thesis Proposal\\ "Monadic Intermediate Language for Modular and Generic Compilers"}
\author{Dmytro Lypai (900620-7113, lypai@student.chalmers.se)}

\begin{document}

\maketitle

Intermediate representation (language) is a data structure built from a source program.
The vast majority of compilers make use of different representations.
Classic examples are: three-address code, Register Transfer Language (RTL), static single assignment (SSA) form and many others.
Compilers can use different representations of a source program during different compilation stages, depending on properties that
these representations have and how suitable they are for a particular stage (for example, easy to produce from a source code,
easy to transform, easy to generate code from etc.).
Most of the transformations and optimisations are performed on intermediate representation(s) of a program.
One of the most important and difficult considerations during different compilation stages are the effects that program fragments
might cause. They have a significant influence on the kinds of transformations that may be performed.

Monads are proved to be a very powerful and generic way to encapsulate and describe different effects
(such as state, exceptions, non-termination etc.). The state of the art example is the Haskell programming language.
Researchers explored monads and monad transformers (a way to combine different monads) for building modular compilers and interpreters,
both as an implementation tool and as an intermediate representation.
There are certain properties of monads (such as monad laws) that enable very interesting program transformations
that may influence the performance and other important properties of the resulting code significantly.

The goals of the thesis are the following:
\begin{itemize}
  \item Design a monadic intermediate language to be used as a compiler intermediate representation.
  \item Explore the suitability of the designed intermediate representation for handling several different source languages.
    The main idea here is to build a compiler for rather different programming languages that support different programming
    paradigms, for example, purely functional and object-oriented, and to evaluate how the same intermediate representation
    can be used to compile, perform analyses and optimisations on such different languages.
    This is particularly important because semantic analysis and optimisations are two major parts of modern compilers.
  \item Explore the opportunities for modular and generic design of the compiler enabled by using the monadic intermediate representation.
    Using such representation should allow building a generic framework for compiler construction which enables
    combining different language features and describing transformations quite easily. Having a common intermediate representation
    also allows a compiler writer to add new source and target languages with much less effort.
\end{itemize}

\end{document}